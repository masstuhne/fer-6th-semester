\documentclass[zavrsnirad]{fer}
% Dodaj opciju upload za generiranje konačne verzije koja se učitava na FERWeb
% Add the option upload to generate the final version which is uploaded to FERWeb


\usepackage{blindtext}


%--- PODACI O RADU / THESIS INFORMATION ----------------------------------------

% Naslov na engleskom jeziku / Title in English
\title{Mobility management of a Raspberry Pi device in a mobile network}

% Naslov na hrvatskom jeziku / Title in Croatian
\naslov{Upravljanje pokretljivosti Raspberry Pi uređaja u mobilnoj mreži}

% Broj rada / Thesis number
\brojrada{1344}

% Autor / Author
\author{Matija Alojz Stuhne}

% Mentor 
\mentor{izv. prof. dr. sc.\@ Marin Vuković}

% Datum rada na engleskom jeziku / Date in English
\date{June, 2024}

% Datum rada na hrvatskom jeziku / Date in Croatian
\datum{lipanj, 2024.}

%-------------------------------------------------------------------------------


\begin{document}


% Naslovnica se automatski generira / Titlepage is automatically generated
\maketitle


%--- ZADATAK / THESIS ASSIGNMENT -----------------------------------------------

% Zadatak se ubacuje iz vanjske datoteke / Thesis assignment is included from external file
% Upiši ime PDF datoteke preuzete s FERWeb-a / Enter the filename of the PDF downloaded from FERWeb
\zadatak{hr_0036540079_73.pdf}


%--- ZAHVALE / ACKNOWLEDGMENT --------------------------------------------------

\begin{zahvale}
  % Ovdje upišite zahvale / Write in the acknowledgment
\end{zahvale}


% Odovud započinje numeriranje stranica / Page numbering starts from here
\mainmatter


% Sadržaj se automatski generira / Table of contents is automatically generated
\tableofcontents


%--- UVOD / INTRODUCTION -------------------------------------------------------
\chapter{Uvod}
\label{pog:uvod}

Neki od radova koje ćemo citirati su \cite{6248073,6247753,ghiglia_pritt_phase_unwrapping,hartley2003multiple,4250461,123DCatch}.
Trebaju nam samo radi testiranja kako izgleda referenciranje rada s konferencije, rada iz časopisa, knjige i Internetske stranice.

\begin{figure}[htb]
  \centering
  \includegraphics[width=0.38\linewidth]{Figures/lenovo_yoga_tab3_pro_front.png} 
  \caption{Moja prva slika}
  \label{slk:prvaslika}
\end{figure}

Referenciramo se na sliku \ref{slk:prvaslika} u sredini rečenice, zatim prije zareza \ref{slk:prvaslika}, te zatim na kraju rečenice \ref{slk:prvaslika}.
Upravo smo testirali radi li naredba \verb|\ref| ispravno u slučaju kada nakon nje slijedi točka.

Sada slijedi jedna jednadžba:
\begin{equation}
  \label{jed:prvajednadzba}
  \int_{-\infty}^{+\infty}f(t)\,dt=F(\omega)
\end{equation}
Jednadžba \eqref{jed:prvajednadzba} je moja prva jednadžba koja defnira par $f(t)\ufrek F(\omega)$ ili $F(\omega)\uvrem f(t)$.


%-------------------------------------------------------------------------------
\chapter{Korišteno sklopovlje}
\label{pog:koristeno_sklopovlje}

\section{Raspberry Pi 5}

\begin{figure}[htb]
  \centering
  \includegraphics[width=0.8\linewidth]{Figures/RaspberryPi5.jpg} 
  \caption{Raspberry Pi 5 \cite{EbenUpton}}
  \label{slk:raspberrypi5}
\end{figure}

\pagebreak

\section{SIM8200EA-M2 5G HAT}
\label{dio:sim8200ea}

SIM8200EA-M2 5G HAT (Hardware Attached on Top) predstavlja dodatak na postojeći Raspberry Pi uređaj, koji omogućava
povezivanje Raspberry Pi uređaja na mobilnu mrežu. Mreže kojima je moguće pristupiti putem HAT-a su: 3G/4G/5G. HAT također
podržava pozicioniranje putem sustava: GPS, GLONASS, Beidou, Galileo, i QZSS.

\begin{figure}[htb]
  \centering
  \includegraphics[width=1\linewidth]{Figures/SIM8200EA-M2-5G-HAT-details-intro.jpg} 
  \caption{SIM8200EA-M2 5G HAT \cite{Waveshare}}
  \label{slk:wavesharesim8200}
\end{figure}

Popis dijelova SIM8200EA-M2 5G HAT-a prikazanog na Slici \ref{slk:wavesharesim8200}:

\begin{enumerate}
  \itemsep-0.5em
  \item Raspberry Pi GPIO zaglavlje
  \item USB3.1 priključak
  \item Audio pogonski sklop
  \item Audio priključak (ulaz)
  \item Prekidač za resetiranje
  \item Prekidač za napajanje
  \item 5V 3A ulaz
  \item Pretvarač napona (sa 5V na 3.3V)
  \item Pretvarač napona (sa 5V na 1.8V)
  \item Pretvarač napona (sa 5V na 4.3V)
  \item Indikator napajanja
  \item Indikator mreže
  \item M.2 priključak
  \item SIM8200EA-M2 modul (detaljnije u \ref{dio:sim8200eamodul})
  \item Priključak za antenu
  \item Priključak za ventilator
  \item Utor za SIM karticu
\end{enumerate}

\pagebreak

\section{SIM8200EA-M2 modul}
\label{dio:sim8200eamodul}

\begin{figure}[htb]
  \centering
  \includegraphics[width=0.8\linewidth]{Figures/SIM8262E-M2_5G_HAT07.png} 
  \caption{Popis antenskih priključaka na modulu te njima dodijeljenih funkcija \cite{WaveshareModule}}
  \label{slk:sim8200eamodul}
\end{figure}

%--- ZAKLJUČAK / CONCLUSION ----------------------------------------------------
\chapter{Zaključak}
\label{pog:zakljucak}

\blindtext


%--- LITERATURA / REFERENCES ---------------------------------------------------

% Literatura se automatski generira iz zadane .bib datoteke / References are automatically generated from the supplied .bib file
% Upiši ime BibTeX datoteke bez .bib nastavka / Enter the name of the BibTeX file without .bib extension
\bibliography{literatura}



%--- SAŽETAK / ABSTRACT --------------------------------------------------------

% Sažetak na hrvatskom
\begin{sazetak}
  Unesite sažetak na hrvatskom.

  \blindtext
\end{sazetak}

\begin{kljucnerijeci}
  prva ključna riječ; druga ključna riječ; treća ključna riječ
\end{kljucnerijeci}


% Abstract in English
\begin{abstract}
  Enter the abstract in English.
  
  \blindtext 
\end{abstract}

\begin{keywords}
  the first keyword; the second keyword; the third keyword
\end{keywords}


%--- PRIVITCI / APPENDIX -------------------------------------------------------

% Sva poglavlja koja slijede će biti označena slovom i riječi privitak / All following chapters will be denoted with an appendix and a letter
\backmatter

\chapter{The Code}

\Blindtext


\end{document}
